 \begin{center}\begin{large} Homework 3, Linear Algebra, Row Echelon Form, Rank, SLE, Invertability, Span\end{large}\end{center}
 \bigskip

\tableofcontents

% 1
\section{Row Echelon Form, Rank}
\begin{problem}%[2 points]
    Bring the following matrices to the Row Echelon Form and find their ranks:

    \begin{enumerate}
        \item[a) ] $C=\begin{bmatrix}3&-2&1\\6&-4&4\end{bmatrix}$,
        
        \item[b) ] $B=\begin{bmatrix}1&4&-2\\2&8&-4\\3&6&2\end{bmatrix}$,
        
        \item[c) ] $C=\begin{bmatrix}1&2&2&1\\3&2&4&2\\-2&-1&-1&-2\end{bmatrix}$,
        
        \item[d) ] $D=\begin{bmatrix}2&4&1&3&2\\-1&-2&1&0&5\\1&6&2&2&2\\3&6&2&5&1\end{bmatrix}$.
    \end{enumerate}
\end{problem}
% \begin{solution}
%      a) $C=\begin{bmatrix}
% 3 & -2 & 1 \\
% 6 & -4 & 4
%     \end{bmatrix}$,
%     \[
%     \begin{bmatrix}
% 3 & -2 & 1 \\
% 6 & -4 & 4
%     \end{bmatrix}
% \xrightarrow{R_2 -2 R_1}
%     \begin{bmatrix}
% 3 & -2 & 1 \\
% 0 & 0 & 2
%     \end{bmatrix}
%     \]
% To bring the matrix to REF it is sufficient to only multiply the first row by $\frac{1}{3}$ and the second row by $\frac{1}{2}$, but we can already say that because there are 2 leading non-zero entries, $\text{rank}(C)=2$.

% For the rest, the technique is the same. You can refer to \href{https://matrixcalc.org/slu.html}{this} and/or \href{https://matrix.reshish.com/rankCalculation.php}{this} webpages for checking your results. 
% \end{solution}

% 2
\section{Systems of linear equations}
\begin{problem}%[2 points]
    Solve the following systems of linear equations using matrices:

    \begin{enumerate}
        \item[a) ] $\begin{cases}
            5x-3y=1\\
            2x+y=18
        \end{cases}$
        
        \item[b) ] $\begin{cases}
            x+y-z=4\\
            3x-y+z=2\\
            x-4z=2
        \end{cases}$
        
        \item[c) ] $\begin{cases}
            x+4y+z=0\\
            x-z=0\\
            9y+7z=0
        \end{cases}$
        
        \item[d) ] $\begin{cases}
            x-y+3z=2\\
            4x+2y-z=-3\\
            -2x-4y+7z=7
        \end{cases}$
        
    \end{enumerate}
\end{problem}

% \begin{solution}
%     b) Let $[\:A\:|\:\vb\:] $ be the augmented matrix
%  \[ [\:A\:|\:\vb\:] = \begin{bmatrix}\

%      1&1&-1&\vert&4\\
%      3&-1&1&\vert&2\\
%      1&0&-4&\vert&2
%  \end{bmatrix} \]
%  Bringing it to REF, we get:
%  \[
%  \begin{bmatrix}

%      1&1&-1&\vert&4\\
%      3&-1&1&\vert&2\\
%      1&0&-4&\vert&2
%  \end{bmatrix} 
%  \xrightarrow{R_2-3R_3}
%  \begin{bmatrix}
%      1&1&-1&\vert&4\\
%      0&-1&13&\vert&-4\\
%      1&0&-4&\vert&2
%  \end{bmatrix} 
%  \xrightarrow{R_1+R_2}\]\[
%  \begin{bmatrix}
%      1&0&12&\vert&0\\
%      0&-1&13&\vert&-4\\
%      1&0&-4&\vert&2
%  \end{bmatrix} 
%  \xrightarrow{R_3-R_1}
%  \begin{bmatrix}
%      1&0&12&\vert&0\\
%      0&-1&13&\vert&-4\\
%      0&0&-16&\vert&2
%  \end{bmatrix} 
%  \xrightarrow{(-1)\cdot R_2}\]\[
%  \begin{bmatrix}
%      1&0&12&\vert&0\\
%      0&1&-13&\vert&4\\
%      0&0&-16&\vert&2
%  \end{bmatrix} 
%  \xrightarrow{-\frac{1}{16}\cdot _3}
%  \begin{bmatrix}
%      1&0&12&\vert&0\\
%      0&1&-13&\vert&4\\
%      0&0&1&\vert&-\frac{1}{8}
%  \end{bmatrix} 
%  \]
%  At this point we can either write it in the form of an SLE and solve for $x,y,z$ or, equivalently, proceed to get the Reduced REF:
%  \[
%  \begin{bmatrix}

%      1&0&12&\vert&0\\
%      0&1&-13&\vert&4\\
%      0&0&1&\vert&-\frac{1}{8}
%  \end{bmatrix} 
%  \xrightarrow{R_1-12R_3}
%  \begin{bmatrix}
%      1&0&0&\vert&\frac{3}{2}\\
%      0&1&-13&\vert&4\\
%      0&0&1&\vert&-\frac{1}{8}
%  \end{bmatrix} 
%  \xrightarrow{R_2+13R_3}
%  \begin{bmatrix}
%      1&0&0&\vert&\frac{3}{2}\\
%      0&1&0&\vert&\frac{19}{8}\\
%      0&0&1&\vert&-\frac{1}{8}
%  \end{bmatrix} 
%  \]
%  whence we have: $x=\dfrac{3}{2},\: y=\dfrac{19}{8},\: z=-\dfrac{1}{8}$.

% For the rest, the technique is the same. You can refer to \href{https://matrixcalc.org/slu.html}{this} webpage for checking your results.
% \end{solution}



% \begin{problem}[2 points]
%     Find, if exists, the inverse of the matrix:

%     \begin{enumerate}
%         \item[a) ] $A=\begin{bmatrix}2&4\\0&1\end{bmatrix}$,
        
%         \item[b) ] $B= \begin{bmatrix}
%             2 & 1 & 3 \\ 5 & -7 & 1 \\ 3 & 0 & -6 
%         \end{bmatrix}$,
        
%         \item[c) ] $C=\begin{bmatrix}0&1&1\\2&-1&1\\1&-1&-1\end{bmatrix}$,
        
%         \item[d) ] $D=\begin{bmatrix}-5&0&3&-1\\0&-5&3&-1\\3&3&-5&4\\-1&-1&4&-6\end{bmatrix}$,

%         \item[e) ] $E=\begin{bmatrix}2&1&4&9\\3&2&4&1\\0&1&-4&2\\-4&-1&-12&-16\end{bmatrix}$.
%     \end{enumerate}

% \end{problem}
% \bigskip

\section{Invertability of a matrix}
\begin{problem}%[1 point]
    Find the value of $a$ such that the following matrix is non-invertible:

        \[ \begin{bmatrix}
            -5 & 0 & a \\ 1 & -2 & 3 \\ 6 & -2 & 1 
        \end{bmatrix} \]
\end{problem}
% \begin{solution}
%     First of all, watch \href{https://www.youtube.com/watch?v=uQhTuRlWMxw}{this} fantastic video by 3blue1brown to form intuition. 

%     A matrix is non-invertible (singular) if and only if if does not have full rank (in our case full-rank is 3). So one approach would be to bring it make the matrix REF and see for which values of $a$ we have that $\text{rank}(a) \neq 3$. Alternatively you can also use the fact that a matrix is non-invertible if and only if it's determinant is zero.

%     Since this whole problem set included REFs and ranks, let's solve this one with determinant approach, so that it's less boring.

%     \[
% \det(\mathbf{A}) = (-5) \det\begin{bmatrix} -2 & 3 \\ -2 & 1 \end{bmatrix} - 
% (0) \det\begin{bmatrix} 1 & 3 \\ 6 & 1 \end{bmatrix} + 
% (a) \det\begin{bmatrix} 1 & -2 \\ 6 & -2 \end{bmatrix}.
% \]
% \[
% \det(\mathbf{A}) = -20 + 10a.  \quad \implies \quad a = 2. \qed
% \]
% \end{solution}

% \bigskip

% \begin{problem}[2 points]
%     Find the eigenvalues and eigenvectors of the following matrices:

%     \begin{enumerate}
%         \item[a) ] $A=\begin{bmatrix}
%             1&2\\0&1\end{bmatrix}$,

%         \item[b) ] $B=\begin{bmatrix}
%             2&1\\1&2
%         \end{bmatrix}$,
        
%         \item[c) ] $C=\begin{bmatrix}
%             6&6&-12\\4&2&-6\\4&3&-7
%         \end{bmatrix}$,
        
%         \item[d) ] $D=\begin{bmatrix}
% -1&0&0\\0&-1&0\\0&0&3        \end{bmatrix}$.
%     \end{enumerate}

%     What are the algebraic and geometric multiplicities of the found eigenvalues?
% \end{problem}

% \bigskip

% \begin{problem}[1 point]
%     Are the following matrices positive definite?

%     \begin{enumerate}
%         \item[a) ] $A=\begin{bmatrix}
%             2&-1&0\\-1&2&-1\\0&-1&2
%         \end{bmatrix}$,

%         \item[b) ] $B=\begin{bmatrix}
%             5&1&1\\1&5&1\\1&1&-2
%         \end{bmatrix}$,

%         \item[c) ] $C=\begin{bmatrix}
%             3&1&2\\1&2&9\\2&2&0
%         \end{bmatrix}$.
%     \end{enumerate}
% \end{problem}


% 4
\section{Dimension of a span}
\begin{problem}%[\textbf{optional, not graded}]
    Find the dimension of the spans of the following vectors:

    \begin{enumerate}
        \item[a) ] $\vv_1=\begin{bmatrix}6\\1\\5\end{bmatrix},  \vv_2=\begin{bmatrix}
            2\\3\\2
        \end{bmatrix}$,

        
        \item[b) ] $\vv_1=\begin{bmatrix}6\\2\end{bmatrix},  \vv_2=\begin{bmatrix}
            1\\3
        \end{bmatrix},  \vv_3=\begin{bmatrix}
            5\\2
        \end{bmatrix}$,
        
        \item[c) ] $\vv_1=\begin{bmatrix}4\\1\\2\end{bmatrix},  \vv_2=\begin{bmatrix}
            -4\\2\\-1
        \end{bmatrix},  \vv_3=\begin{bmatrix}
            8\\-1\\7
        \end{bmatrix}$,
        
        \item[d) ] $\vv_1=\begin{bmatrix}3\\-2\\1\end{bmatrix},  \vv_2=\begin{bmatrix}
            6\\-4\\4
        \end{bmatrix}$.
        
    \end{enumerate}
\end{problem}
% \begin{solution}
%     To find the dimension of the span of a set of vectors, we calculate the \textbf{rank} of the matrix formed by stacking these vectors as columns. The rank corresponds to the number of linearly independent vectors, which is equal to the dimension of the span.

% \begin{enumerate}
%     \item 
%     Form the matrix \(\mathbf{A}\) by placing the vectors as columns:
%     \[
%     \mathbf{A} = \begin{bmatrix}
%        6 & 2 \\
%        1 & 3 \\
%        5 & 2
%     \end{bmatrix}.
%     \]

%     \item From this point this is the same exercise as the first one :)
% \end{enumerate}
% \end{solution}
