 \begin{center}\begin{large} Homework Problems 1 \end{large}\end{center}
 \bigskip

% \documentclass{article}
% \usepackage{amsmath}

% \begin{document}


\begin{problem}%[1 point]
    Add the following vectors:

    \begin{enumerate}
        \item[a) ] $\va+\vb$, where $\va= \begin{bmatrix} 3 \\ -1 \\ 2 \end{bmatrix}, \vb=\begin{bmatrix} 2 \\ 4 \\ -5 \end{bmatrix}$,
        
        \item[b) ] $\vv^T+\vu^T+\vw^T$, where $\vv= \begin{bmatrix} 1 \\ -2 \\ 0 \end{bmatrix}, \vu = \begin{bmatrix} -3 \\ 5 \\ 2.5 \end{bmatrix}, \vw= \begin{bmatrix} -2 \\ 3 \\ 2 \end{bmatrix}$. 
    \end{enumerate}
\end{problem}

\begin{sol} 
    \subsubsection*{Part (a)}

    \[
    \va + \vb = \begin{bmatrix} 3+2 \\ -1+4 \\ 2+(-5) \end{bmatrix} = \textcolor{orange}{\begin{bmatrix} 5 \\ 3 \\ -3 \end{bmatrix}}
    \]
    
    \subsubsection*{Part (b)}
    
    \[
    \vv^T + \vu^T + \vw^T = \begin{bmatrix} 1 & -2 & 0 \end{bmatrix} + \begin{bmatrix} -3 & 5 & 2.5 \end{bmatrix} + \begin{bmatrix} -2 & 3 & 2 \end{bmatrix}
    \]
    \[
    = \begin{bmatrix} 1+(-3)+(-2) & -2+5+3 & 0+2.5+2 \end{bmatrix} = \textcolor{orange}{\begin{bmatrix} -4 & 6 & 4.5 \end{bmatrix}}
    \]
\end{sol}


\begin{problem}%[1 point]
    Multiply the following vectors and scalars:

    \begin{enumerate}
        \item[a) ] $2 \cdot \vu$,  where $\vu= \begin{bmatrix} 3 \\ -1 \\ 2 \end{bmatrix} $,
        
        \item[b) ] $ \vy \cdot (-3.2)$, where $\vy = \begin{bmatrix} 1 \\ -2 \\ 0 \end{bmatrix}$,
        
        \item[c) ] $0 \cdot \vz$, where $\vz = \begin{bmatrix}4 \\ 0 \\ -7 \end{bmatrix}$.
    \end{enumerate}
\end{problem}

\begin{sol}
    \subsubsection*{Part (a)}
    \[
    2 \cdot \vu = 2 \cdot \begin{bmatrix} 3 \\ -1 \\ 2 \end{bmatrix} = \begin{bmatrix} 2 \cdot 3 \\ 2 \cdot (-1) \\ 2 \cdot 2 \end{bmatrix} = \textcolor{orange}{\begin{bmatrix} 6 \\ -2 \\ 4 \end{bmatrix}}
    \]
    
    \subsubsection*{Part (b)}
    \[
    \vy \cdot (-3.2) = (-3.2) \cdot \begin{bmatrix} 1 \\ -2 \\ 0 \end{bmatrix} = \begin{bmatrix} -3.2 \cdot 1 \\ -3.2 \cdot (-2) \\ -3.2 \cdot 0 \end{bmatrix} = \textcolor{orange}{\begin{bmatrix} -3.2 \\ 6.4 \\ 0 \end{bmatrix}}
    \]
    
    \subsubsection*{Part (c)}
    \[
    0 \cdot \vz = 0 \cdot \begin{bmatrix} 4 \\ 0 \\ -7 \end{bmatrix} = \begin{bmatrix} 0 \cdot 4 \\ 0 \cdot 0 \\ 0 \cdot (-7) \end{bmatrix} = \textcolor{orange}{\begin{bmatrix} 0 \\ 0 \\ 0 \end{bmatrix}}
    \]
\end{sol}

%%%%%%%%%%%%%%%%%%%%%% 3 %%%%%%%%%%%%%%%%%%
\begin{problem}%[2 points]
    Calculate the dot product:
    
a) $\vu \cdot \vv$, where $\vu=\begin{bmatrix}3\\-1\\2 \end{bmatrix}$, $\vv=\begin{bmatrix}2\\4\\-5 \end{bmatrix}$,
\medskip

b) $\vx \cdot \vy$, where $\vx =\begin{bmatrix}2\\7\\3 \end{bmatrix}$, $\vy =-\vx$.
\end{problem}

\begin{sol}
        \subsubsection*{Part (a)}
    \[
    \vu \cdot \vv = \begin{bmatrix} 3 \\ -1 \\ 2 \end{bmatrix} \cdot \begin{bmatrix} 2 \\ 4 \\ -5 \end{bmatrix} = 3 \cdot 2 + (-1) \cdot 4 + 2 \cdot (-5)
    \]
    \[
    = 6 - 4 - 10 = \textcolor{orange}{-8}
    \]
    
    \subsubsection*{Part (b)}
    \[
    \vx \cdot \vy = \vx \cdot (-\vx) = -(\vx \cdot \vx)
    \]
    \[
    \vx \cdot \vx = \begin{bmatrix} 2 \\ 7 \\ 3 \end{bmatrix} \cdot \begin{bmatrix} 2 \\ 7 \\ 3 \end{bmatrix} = 2 \cdot 2 + 7 \cdot 7 + 3 \cdot 3
    \]
    \[
    = 4 + 49 + 9 = 62
    \]
    \[
    \vx \cdot \vy = -62 = \textcolor{orange}{-62}
    \]
\end{sol}



\newpage
\begin{problem}%[3 points]
    Given the vectors $\mathbf{u} = \begin{bmatrix} 4 \\ -1 \\ 6 \end{bmatrix}$ and $\mathbf{v} = \begin{bmatrix} 5 \\ 1 \\ 2 \end{bmatrix}$, calculate:
        \begin{enumerate}
            \item[a) ] $\vu+\vv$
            \item[b) ] $3\vu-2\vv$
            \item[c) ] $\vu \cdot \vv$
            \item[d) ] $(3\vv - 6\vu) \cdot \vv$
            \item[e) ] $\vv \cdot (4\vv^T - \vu^T)^T$
        \end{enumerate}
\end{problem}

\begin{sol}
        \subsubsection*{Part (a)}
    \[
    \vu + \vv = \begin{bmatrix} 4 \\ -1 \\ 6 \end{bmatrix} + \begin{bmatrix} 5 \\ 1 \\ 2 \end{bmatrix} = \begin{bmatrix} 4+5 \\ -1+1 \\ 6+2 \end{bmatrix} = \textcolor{orange}{\begin{bmatrix} 9 \\ 0 \\ 8 \end{bmatrix}}
    \]
    
    \subsubsection*{Part (b)}
    \[
    3\vu - 2\vv = 3 \cdot \begin{bmatrix} 4 \\ -1 \\ 6 \end{bmatrix} - 2 \cdot \begin{bmatrix} 5 \\ 1 \\ 2 \end{bmatrix}
    \]
    \[
    = \begin{bmatrix} 12 \\ -3 \\ 18 \end{bmatrix} - \begin{bmatrix} 10 \\ 2 \\ 4 \end{bmatrix} = \begin{bmatrix} 12-10 \\ -3-2 \\ 18-4 \end{bmatrix} = \textcolor{orange}{\begin{bmatrix} 2 \\ -5 \\ 14 \end{bmatrix}}
    \]
    
    \subsubsection*{Part (c)}
    \[
    \vu \cdot \vv = \begin{bmatrix} 4 \\ -1 \\ 6 \end{bmatrix} \cdot \begin{bmatrix} 5 \\ 1 \\ 2 \end{bmatrix} = 4 \cdot 5 + (-1) \cdot 1 + 6 \cdot 2
    \]
    \[
    = 20 - 1 + 12 = \textcolor{orange}{31}
    \]
    
    \subsubsection*{Part (d)}
    \[
    (3\vv - 6\vu) \cdot \vv = \left(3 \cdot \begin{bmatrix} 5 \\ 1 \\ 2 \end{bmatrix} - 6 \cdot \begin{bmatrix} 4 \\ -1 \\ 6 \end{bmatrix}\right) \cdot \begin{bmatrix} 5 \\ 1 \\ 2 \end{bmatrix}
    \]
    \[
    = \left(\begin{bmatrix} 15 \\ 3 \\ 6 \end{bmatrix} - \begin{bmatrix} 24 \\ -6 \\ 36 \end{bmatrix}\right) \cdot \begin{bmatrix} 5 \\ 1 \\ 2 \end{bmatrix}
    \]
    \[
    = \begin{bmatrix} -9 \\ 9 \\ -30 \end{bmatrix} \cdot \begin{bmatrix} 5 \\ 1 \\ 2 \end{bmatrix} = -9 \cdot 5 + 9 \cdot 1 + (-30) \cdot 2
    \]
    \[
    = -45 + 9 - 60 = \textcolor{orange}{-96}
    \]
    
    \subsubsection*{Part (e)}
    \[
    \vv \cdot (4\vv^T - \vu^T)^T = \vv \cdot (4 \cdot \vv - \vu)
    \]
    \[
    = \begin{bmatrix} 5 \\ 1 \\ 2 \end{bmatrix} \cdot \left(4 \cdot \begin{bmatrix} 5 \\ 1 \\ 2 \end{bmatrix} - \begin{bmatrix} 4 \\ -1 \\ 6 \end{bmatrix}\right)
    \]
    \[
    = \begin{bmatrix} 5 \\ 1 \\ 2 \end{bmatrix} \cdot \left(\begin{bmatrix} 20 \\ 4 \\ 8 \end{bmatrix} - \begin{bmatrix} 4 \\ -1 \\ 6 \end{bmatrix}\right)
    \]
    \[
    = \begin{bmatrix} 5 \\ 1 \\ 2 \end{bmatrix} \cdot \begin{bmatrix} 16 \\ 5 \\ 2 \end{bmatrix} = 5 \cdot 16 + 1 \cdot 5 + 2 \cdot 2
    \]
    \[
    = 80 + 5 + 4 = \textcolor{orange}{89}
    \]
\end{sol}

\medskip

\begin{problem}%[1 point]
    A translation office translated $\va=[24, 17, 9, 13]$ documents from English, French, German and Russian, respectively. For each of those languages, it takes about $\vb=[5, 10, 11, 7]$ minutes to translate one page.
    
    How much time did they spend translating in total? How much did each of the translators spend on average if there are $4$ translators in the office? Write an expression for this amount in terms of the vectors $\va$ and $\vb$.
\end{problem}

\begin{sol}
    \subsubsection*{Total Time Spent}

    The total time spent translating is the dot product of the vectors \(\va\) and \(\vb\):
    \[
    \text{Total Time} = \va \cdot \vb = \begin{bmatrix} 24 \\ 17 \\ 9 \\ 13 \end{bmatrix} \cdot \begin{bmatrix} 5 \\ 10 \\ 11 \\ 7 \end{bmatrix}
    \]
    \[
    = (24 \cdot 5) + (17 \cdot 10) + (9 \cdot 11) + (13 \cdot 7)
    \]
    \[
    = 120 + 170 + 99 + 91 = \textcolor{orange}{480 \text{ minutes}}.
    \]
    
    \subsubsection*{Average Time per Translator}
    
    The average time per translator, given there are \(4\) translators, is:
    \[
    \text{Average Time} = \frac{\text{Total Time}}{4} = \frac{480}{4} = \textcolor{orange}{120 \text{ minutes}}.
    \]
    
    \subsubsection*{Expression in Terms of \(\va\) and \(\vb\)}
    
    The total time in terms of \(\va\) and \(\vb\) is:
    \[
    \text{Total Time} = \va \cdot \vb
    \]
    
    The average time per translator is:
    \[
    \text{Average Time} = \frac{\va \cdot \vb}{4}
    \]
\end{sol}


\medskip

\begin{problem}[\textbf{additional}]%[2 points]
    Show that the following set is a vector space:

    \begin{enumerate}
        \item [a)] $A = \R$,
        
        \item [b)] $B = \bigg\{ \begin{bmatrix} a \\ 0 \end{bmatrix} \mid \text{ for all numbers }a\in\R\bigg\}$,

        \item [c)] The set of all polynomials of the form $ax+b$ \\(i.e. all polynomials of degree $\le 1$). 

    \end{enumerate}
    \bigskip
    {\small \textit{Hint:} \textit{to show that a set is a vector space, check that the sum of any two elements is again in the set, and that if we multiply any element of the set with any number, the result will again be in the set. Then check that conditions 1-8 (in the lecture) hold.}}
\end{problem}
\medskip
\begin{problem}[\textbf{additional}]%[2 points]
    Show that the following set is \textit{not} a vector space:

    \begin{enumerate}
        \item [a)] $A = \N$,
        
        \item [b)] $B =  \bigg\{ \begin{bmatrix} a \\ 1 \end{bmatrix} \mid \text{ for all numbers }a\in\R\bigg\}$,
        
        \item [c)] The set of all polynomials of the form $ax+b$ where $a \ne 0$.

    \end{enumerate}
    \bigskip
    {\small \textit{Hint:} \textit{Show that if we sum two elements or multiply an element with some number, the result might not always belong to the set, or that some of the conditions 1-8 does not hold.}}
\end{problem}


        
        