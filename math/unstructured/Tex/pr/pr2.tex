 \begin{center}\begin{large} Practice Problems 2
 \end{large}\end{center}
 \bigskip

% \documentclass{article}
% \usepackage{amsmath}

% \begin{document}


% \begin{problem}
%     Find the trace and determinant of the following matrices:

%     \begin{enumerate}
%         \item[a) ] $A=\begin{bmatrix}
%         2&-1\\-2&-5   \end{bmatrix}$,
        
%         \item[b) ] $B=\begin{bmatrix}
%         4&10\\34&0   \end{bmatrix}$,
        
%         \item[c) ]  $C=\begin{bmatrix}
%         2&6&3\\9&-1&4\\1&5&-2   \end{bmatrix}$,

        
%         \item[d) ]  $D=\begin{bmatrix}
%         5&-1&2\\3&1&-1\\-1&-1&-1   \end{bmatrix}$,
        
%         \item[e) ] $E=\begin{bmatrix}
%         2&0&2&4\\2&0&2&5\\2&0&2&6\\-2&0&-2&-7  \end{bmatrix}$.
%     \end{enumerate}
% \end{problem}
% \bigskip

% \begin{problem}
%     Find the inverse matrix:
%     \begin{enumerate}
%         \item[a) ] $A=\begin{bmatrix}
%         4&3\\5&-1   \end{bmatrix}$,

%         \item[b) ] $B=\begin{bmatrix}
%         7&10\\-3&2   \end{bmatrix}$.
%     \end{enumerate}
% \end{problem}
% \bigskip
% % % \bigskip
% % \newpage

% \begin{center}
%      \begin{large}
%          Solutions
%      \end{large}
%  \end{center}

%  \bigskip



%  \begin{solution} b)

%      \[ 5\va - 10\vb = 5\cdot (\va-2\vb) = 5\cdot \bigg( \begin{bmatrix} 17 \\ 24 \end{bmatrix}-2\cdot\begin{bmatrix} 5 \\ 12 \end{bmatrix} \bigg) =5\cdot \begin{bmatrix} 17-2\cdot 5 \\ 24-2\cdot 12 \end{bmatrix}=5\cdot \begin{bmatrix} 7 \\ 0 \end{bmatrix} =\begin{bmatrix} 35 \\ 0 \end{bmatrix} \]
%  \end{solution}

% \begin{solution} b)
%      \[ \vx \cdot \vy = -(\vx\cdot\vx) = -(2\cdot2+7\cdot 7 + 3\cdot 3) = -62 \]
%  \end{solution}

% \begin{solution}
%  b) $ \va \cdot \vb$ shows the area of lawn mowed by Michael, in square feet,

% c) $0.5 \va \cdot \vb.$
% \end{solution}

% \begin{solution} 
% \begin{enumerate}
%      \item[a) ]  $A = \Z$ is not a vector space; for example, $1 \in B$ but for the scalar $c=0.2$, $c \cdot 1 = 0.2 \not\in B$,
%      \item[b) ]  $B$ is a vector space (check the axioms),
%      \item[c) ]  $C$ is not a vector space, as the axiom 7 does not hold:
%  \[ c\cdot\bigg( \begin{bmatrix} x_1 \\ x_2\end{bmatrix} +  \begin{bmatrix} y_1 \\ y_2\end{bmatrix}\bigg) \neq c \cdot  \begin{bmatrix} x_1 \\ x_2\end{bmatrix}  + c \cdot  \begin{bmatrix} y_1 \\ y_2\end{bmatrix}.  \]
%      \item[d) ]  $D$ is a vector space (check the axioms).
%  \end{enumerate}

%  \end{solution}

% % % 4
%  \begin{solution} b)

%      \[ \va+\vb = \begin{bmatrix} 12-1 \\ 5+2 \\ 0-2 \end{bmatrix} = \begin{bmatrix} 11\\ 7 \\ -2 \end{bmatrix} \]
%  \[ \|\va+\vb\|_1 = |11|+|7|+|-2|=11+7+2=20  \]
%  \[ \|\va+\vb\|_2 = \sqrt{11^2+7^2+(-2)^2}=\sqrt{174}  \]

%  c)

%     \[ \|\vc\|_1 = |3|+|4|=7  \]
%     \[ \|13\vc\|_1 = 13\cdot \|\vc\|_1 = 13\cdot7=91  \]
%     \smallskip
%     \[ \|\vc\|_2 = \sqrt{3^2+4^2}=5  \]
%     \[ \|13\vc\|_2 = 13\cdot \|\vc\|_2 = 13\cdot 5=65  \]
%  \end{solution}
%  \medskip
%  \begin{solution} b) Let $\alpha$ be the angle between $\va$ and $\vb$. Then,

%  \[\cos{\alpha} = \frac{\va\cdot \vb}{\|\va\|\cdot\|\vb\|} = \frac{3 \cdot 3 + 3 \cdot 0}{\sqrt{3^2 + 0^2}\cdot \sqrt{3^2 + 3^2}} = \frac{9}{3\cdot 3\sqrt{2}} = \frac{1}{ \sqrt{2}} \]

%  \[\alpha = \arccos\frac{1}{ \sqrt{2}} = 45^{\circ}.\]
%  \end{solution}
%  \begin{solution} c) 

%  \[A - B = \begin{bmatrix}
%          2-2&5-1&4-5\\-3-(-5)&-2-2&4-2\\5-1&9-6&2-(-1)   \end{bmatrix}
%          = \begin{bmatrix}
%          0&4&-1\\2&-4&2\\4&3&3   \end{bmatrix}
%          \]

%  \begin{align*}
%      (A-&B)C=\begin{bmatrix}
%          0&4&-1\\2&-4&2\\4&3&3   \end{bmatrix} \begin{bmatrix}
%          4&-1\\1&2\\3&3   \end{bmatrix} \\&= \begin{bmatrix}
%          (0 \cdot 4 + 4 \cdot1+(-1) \cdot3) &  (0  \cdot (-1) + 4  \cdot 2 + (-1)  \cdot 3 \\
%          (2 \cdot 4 + (-4) \cdot1+2 \cdot3) &  (2  \cdot (-1) + (-4)  \cdot 2 + 2  \cdot 3 \\
%             (4 \cdot 4 + 3 \cdot1+3 \cdot3) &  (4 \cdot (-1) + 3  \cdot 2 + 3  \cdot 3 \end{bmatrix}
%             = \begin{bmatrix}
%          1&  5\\
%          10& -4 \\
%             28& 11\end{bmatrix}
%  \end{align*}
        
%  \end{solution}
%  \medskip
%  \begin{solution} a)

%  \[\text{tr}(A)=\text{tr}\bigg(\begin{bmatrix}
%          7&-1&5\\6&8&0\\10&2&-6   \end{bmatrix}\bigg) = 7+8+(-6)=9\]

%      b)

%  \[\text{det}(A)=\begin{vmatrix} 7&4\\-3&-4  \end{vmatrix}=7\cdot (-4) - 4\cdot(-3)=-28+12=-16\]


%      d) Using the cofactor method on the 4th column,

%  \begin{align*}
%      \text{det}(A)=\begin{vmatrix} 7&2&4&0\\1&5&0&0\\4&2&0&1\\-2&3&1&2  \end{vmatrix}&=0\cdot (-1)^{1+4}\cdot\begin{vmatrix} 1&5&0\\4&2&0\\-2&3&1  \end{vmatrix}  + 0\cdot (-1)^{2+4}\cdot\begin{vmatrix} 7&2&4\\4&2&0\\-2&3&1 \end{vmatrix}\\&  + 1\cdot (-1)^{3+4}\cdot\begin{vmatrix} 7&2&4\\1&5&0\\-2&3&1 \end{vmatrix} + 2\cdot (-1)^{4+4}\cdot\begin{vmatrix} 7&2&4\\1&5&0\\4&2&0 \end{vmatrix}\\&=-\begin{vmatrix} 7&2&4\\1&5&0\\-2&3&1 \end{vmatrix} +2\cdot\begin{vmatrix} 7&2&4\\1&5&0\\4&2&0 \end{vmatrix}\\&=-85+2\cdot(-72)=-229
%  \end{align*}

%      e) $\det(AB)=\det(A) \cdot \det(B) = \det(A) \cdot 0 = 0.$

%  \end{solution}

%  \begin{solution}
%      b)
%      \[ \det(A) = \begin{vmatrix}
%          7&10\\-3&2
%      \end{vmatrix} = 7\cdot 2 - 10\cdot(-3) = 14 + 30 = 44 \ne 0 \]
%      Therefore, $A$ is invertible and
%      \[ A^{-1} = \frac{1}{44}\begin{bmatrix}
%          2&-10\\3&7
%      \end{bmatrix} \]
%  \end{solution}


\begin{problem}%[2 points]
    Bring the following matrices to the Row Echelon Form and find their ranks:

    \begin{enumerate}
        \item[a) ] $A=\begin{bmatrix}6&2\\1&3\\5&2\end{bmatrix}$,
        
        \item[b) ] $B=\begin{bmatrix}4&1&2\\-4&2&-1\\8&-1&7\end{bmatrix}$,
        
        \item[c) ] $C=\begin{bmatrix}3&-2&1\\6&-4&4\end{bmatrix}$.

        \item[d) ] $D=\begin{bmatrix}2&1&4&9\\3&2&4&1\\0&1&-4&2\\-4&-1&-12&-16\end{bmatrix}$,
        
    \end{enumerate}
\end{problem}

\bigskip


\begin{problem}
    Check whether the vectors $\vv_1, \vv_2, \vv_3$ are linearly independent if:

    \begin{enumerate}
        \item[a) ] $\vv_1=\begin{bmatrix}
            1 \\ 4 \\ 2
        \end{bmatrix}, \vv_2=\begin{bmatrix}
            3 \\ 5 \\ 1
        \end{bmatrix}, \vv_3=\begin{bmatrix}
            1 \\ -3 \\ -3
        \end{bmatrix}, 
        $
        
        \item[b) ] $\vv_1=\begin{bmatrix}
            2 \\ 1 \\ -2
        \end{bmatrix}, \vv_2=\begin{bmatrix}
            3 \\ 2 \\ 0
        \end{bmatrix}, \vv_3=\begin{bmatrix}
            0 \\ 1 \\ 3
        \end{bmatrix}.
        $
        
    \end{enumerate}
\end{problem}

\bigskip 
% \newpage
\begin{problem}
    Solve the following systems of linear equations:

    \begin{enumerate}
        \item[a) ] $\begin{cases}
            3x-y=4\\
            2x+5y=-3
        \end{cases}$
        
        \item[b) ] $\begin{cases}
            x+y-z=4\\
            3x-y+z=2\\
            x-4z=2
        \end{cases}$
        
        \item[c) ] $\begin{cases}
            2x-y+z=5\\
            4x-3y=9\\
            2x-2y-z=4\\
        \end{cases}$
        
    \end{enumerate}
\end{problem}
\bigskip

% \begin{problem}
%     Find, if exists, the inverse of the matrix:

%     \begin{enumerate}
%         \item[a) ] $A=\begin{bmatrix}1&-1\\0&2\end{bmatrix}$,
        
%         \item[b) ] $B=\begin{bmatrix}1&-1&1\\2&3&0\\0&-2&1\end{bmatrix}$,
        
%         \item[c) ] $C=\begin{bmatrix}2&3&1\\3&3&1\\2&4&1\end{bmatrix}$,
        
%         \item[d) ] $D=\begin{bmatrix}2&-17&11\\-1&11&-7\\0&3&-2\end{bmatrix}$.
%     \end{enumerate}
% \end{problem}
% \bigskip

% \begin{problem}
%     Find the eigenvalues and eigenvectors of the following matrices:

%     \begin{enumerate}
%         \item[a) ] $A=\begin{bmatrix}
%             -5&2\\-7&4
%         \end{bmatrix}$,
        
%         \item[b) ] $B=\begin{bmatrix}
%             1&0\\0&1
%         \end{bmatrix}$,
        
        
%         \item[c) ] $C=\begin{bmatrix}
%             2&2&-2\\1&3&-1\\-1&1&1
%         \end{bmatrix}$,
%     \end{enumerate}

%     What are the algebraic and geometric multiplicities of the found eigenvalues?
% \end{problem}

% % \bigskip


 \newpage



\begin{solution}
    c) $C=\begin{bmatrix}
3 & -2 & 1 \\
6 & -4 & 4
    \end{bmatrix}$,
    \[
    \begin{bmatrix}
3 & -2 & 1 \\
6 & -4 & 4
    \end{bmatrix}
\xrightarrow{R_2 -2 R_1}
    \begin{bmatrix}
3 & -2 & 1 \\
0 & 0 & 2
    \end{bmatrix}
    \]
To bring the matrix to REF it is sufficient to only multiply the first row by $\frac{1}{3}$ and the second row by $\frac{1}{2}$, but we can already say that because there are 2 leading non-zero entries, $\text{rank}(C)=2$.

    
\end{solution}
\bigskip

\begin{solution}
b) To check if a set of vectors is linearly independent or not, we can put together those vectors as rows or columns of a matrix and find its rank. For example, 
\[B=
\begin{bmatrix}
     \vv_1^T \\ \vv_2^T \\ \vv_3^T
\end{bmatrix}
 =
 \begin{bmatrix}
     2 & 1 & -2 \\
     3 & 2 & 0 \\
     0 & 1 & 3
 \end{bmatrix}
 \xrightarrow{R_2 -1.5 R_1}
 \begin{bmatrix}
     2 & 1 & -2 \\
     0 & -0.5 & 3 \\
     0 & 1 & 3
 \end{bmatrix}
 \xrightarrow[R_3 + 2R_2]{R_1+2R_2}
 \begin{bmatrix}
     2 & 0 & 4 \\
     0 & -0.5 & 3 \\
     0 & 0 & 9
 \end{bmatrix}
 \]

 Again, multiplying each row with a corresponding number will result in 3 leading ones, therefore $\operatorname{rank}(B)=3$ which simply means that its rows, $\vv_1,\vv_2,$ and $\vv_3$, are linearly independent.

 \end{solution}
 \bigskip

 \begin{solution}
 b) Let $[\:A\:|\:\vb\:] $ be the augmented matrix
 \[ [\:A\:|\:\vb\:] = \begin{bmatrix}\

     1&1&-1&\vert&4\\
     3&-1&1&\vert&2\\
     1&0&-4&\vert&2
 \end{bmatrix} \]
 Bringing it to REF, we get:
 \[
 \begin{bmatrix}

     1&1&-1&\vert&4\\
     3&-1&1&\vert&2\\
     1&0&-4&\vert&2
 \end{bmatrix} 
 \xrightarrow{R_2-3R_3}
 \begin{bmatrix}
     1&1&-1&\vert&4\\
     0&-1&13&\vert&-4\\
     1&0&-4&\vert&2
 \end{bmatrix} 
 \xrightarrow{R_1+R_2}\]\[
 \begin{bmatrix}
     1&0&12&\vert&0\\
     0&-1&13&\vert&-4\\
     1&0&-4&\vert&2
 \end{bmatrix} 
 \xrightarrow{R_3-R_1}
 \begin{bmatrix}
     1&0&12&\vert&0\\
     0&-1&13&\vert&-4\\
     0&0&-16&\vert&2
 \end{bmatrix} 
 \xrightarrow{(-1)\cdot R_2}\]\[
 \begin{bmatrix}
     1&0&12&\vert&0\\
     0&1&-13&\vert&4\\
     0&0&-16&\vert&2
 \end{bmatrix} 
 \xrightarrow{-\frac{1}{16}\cdot _3}
 \begin{bmatrix}
     1&0&12&\vert&0\\
     0&1&-13&\vert&4\\
     0&0&1&\vert&-\frac{1}{8}
 \end{bmatrix} 
 \]
 At this point we can either write it in the form of an SLE and solve for $x,y,z$ or, equivalently, proceed to get the Reduced REF:
 \[
 \begin{bmatrix}

     1&0&12&\vert&0\\
     0&1&-13&\vert&4\\
     0&0&1&\vert&-\frac{1}{8}
 \end{bmatrix} 
 \xrightarrow{R_1-12R_3}
 \begin{bmatrix}
     1&0&0&\vert&\frac{3}{2}\\
     0&1&-13&\vert&4\\
     0&0&1&\vert&-\frac{1}{8}
 \end{bmatrix} 
 \xrightarrow{R_2+13R_3}
 \begin{bmatrix}
     1&0&0&\vert&\frac{3}{2}\\
     0&1&0&\vert&\frac{19}{8}\\
     0&0&1&\vert&-\frac{1}{8}
 \end{bmatrix} 
 \]
 whence we have: $x=\dfrac{3}{2},\: y=\dfrac{19}{8},\: z=-\dfrac{1}{8}$.
 \end{solution}

 \bigskip

 \begin{solution}
     c) \[
 \begin{bmatrix}
     2 & 3 & 1 & | & 1 & 0 & 0 \\
     3 & 3 & 1 & | & 0 & 1 & 0 \\
     2 & 4 & 1 & | & 0 & 0 & 1 \\
 \end{bmatrix} \xrightarrow{R_1\leftrightarrow R_2} \begin{bmatrix}
     3 & 3 & 1 & | & 0 & 1 & 0 \\
     2 & 3 & 1 & | & 1 & 0 & 0 \\
     2 & 4 & 1 & | & 0 & 0 & 1 \\
 \end{bmatrix} \xrightarrow{R_1- R_2}
 \begin{bmatrix}
     1 & 0 & 0 & | & -1 & 1 & 0 \\
     2 & 3 & 1 & | & 1 & 0 & 0 \\
     2 & 4 & 1 & | & 0 & 0 & 1 \\
 \end{bmatrix}
 \]
 \[
 \xrightarrow{R_3- R_2}\begin{bmatrix}
     1 & 0 & 0 & | & -1 & 1 & 0 \\
     2 & 3 & 1 & | & 1 & 0 & 0 \\
     0 & 1 & 0 & | & -1 & 0 & 1 \\
 \end{bmatrix}\xrightarrow{R_3 \leftrightarrow R_2}
 \begin{bmatrix}
     1 & 0 & 0 & | & -1 & 1 & 0 \\  
     0 & 1 & 0 & | & -1 & 0 & 1 \\
     2 & 3 & 1 & | & 1 & 0 & 0 \\
    
 \end{bmatrix}\]\[\xrightarrow{R_3 -2 R_1}
 \begin{bmatrix}
     1 & 0 & 0 & | & -1 & 1 & 0 \\  
     0 & 1 & 0 & | & -1 & 0 & 1 \\
     0 & 3 & 1 & | & 3 & -2 & 0 \\    
 \end{bmatrix}\xrightarrow{R_3 -3 R_2}
 \begin{bmatrix}
     1 & 0 & 0 & | & -1 & 1 & 0 \\  
     0 & 1 & 0 & | & -1 & 0 & 1 \\
     0 & 0 & 1 & | & 6 & -2 & -3 \\    
 \end{bmatrix}%=[\:I\:|\:A^{-1}\:]
 \]

 Thus,
 \[ C^{-1} = \begin{bmatrix} -1 & 1 & 0 \\ -1 & 0 & 1 \\ 6 &-2 & -3 \end{bmatrix} \]

 \end{solution}

 \bigskip
 \begin{solution}
     a) $A=\begin{bmatrix}
             -5&2\\-7&4
         \end{bmatrix}$,
        
        
         We write its characteristic polynomial to find the eigenvalues:
         \[  \det(A-\lambda I)=\left|\begin{array}{cc}
             -5-\lambda&2\\-7&4-\lambda
         \end{array}\right| =\lambda^2+\lambda-6=(\lambda+3)(\lambda-2)=0, \]
         hence both roots $\lambda = -3$ and $\lambda = 2$ have algebraic multiplicities of $1$.
         \begin{itemize}
             \item To find the eigenvectors $\vv=\begin{bmatrix}
             v_1\\v_2
         \end{bmatrix}\ne \textbf{0}$ corresponding to $\lambda = -3$, we let:
         \[
         A\vv=-3 \vv
         \]
         \[
         \begin{bmatrix}
             -5v_1 + 2v_2 \\ -7v_1 + 4v_2
         \end{bmatrix} = \begin{bmatrix}
             -3v_1 \\ -3v_2
         \end{bmatrix}
         \]
         whence we have $v_1=v_2$ (in fact, both equations give the same result). So the eigenvectors are:
         \[
         E_{-3} = \left\{ a \cdot \begin{bmatrix}
             1 \\ 1
         \end{bmatrix} \bigg\vert\: a\in\R \right\}
         \]

         Due to the fact that all vectors in $E_{-3}$ are multiples of only one vector $\begin{bmatrix}
             1 \\ 1
         \end{bmatrix}$, we see that $\dim(E_{-3})=1$, so $\lambda=-3$ has geometric multiplicity of $1$.

        
             \item To find the eigenvectors corresponding to $\lambda = 2$, we let:
         \[
         A\vv=2 \vv
         \]
         \[
         \begin{bmatrix}
             -5v_1 + 2v_2 \\ -7v_1 + 4v_2
         \end{bmatrix} = \begin{bmatrix}
             2v_1 \\ 2v_2
         \end{bmatrix}
         \]
         whence $v_1=\frac{2}{7}v_2$ (in fact, both equations give the same result). So the eigenvectors are:
         \[
         E_{2} = \left\{ a \cdot \begin{bmatrix}
             \frac{2}{7} \\ 1
         \end{bmatrix} \bigg\vert\: a\in\R \right\} = \left\{ a \cdot \begin{bmatrix}
             2 \\ 7
         \end{bmatrix} \bigg\vert\: a\in\R \right\}
         \]

         With the same logic, $\dim(E_{2})=1$, so $\lambda=2$ has geometric multiplicity of $1$ too.

         \end{itemize}
        
 \end{solution}