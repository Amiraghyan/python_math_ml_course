 \begin{center}\begin{large} Practice Problems 4
 \end{large}\end{center}
 \bigskip

\begin{problem}
    Calculate $\displaystyle \int f(x)\, dx$:
    \begin{enumerate}
        \item[a) ] $f(x)=3x^2$,
        \item[b) ] $f(x)=x+6\cos x$,
        \item[c) ] $f(x)=\dfrac{3}{x}-1$,
        \item[d) ] $f(x)=\dfrac{4x}{1-2x^2}$,
        \item[e) ] $f(x)=\dfrac{2x}{1-x^2}$,
        \item[f) ] $f(x)=\text{tg}\,x$.
    \end{enumerate}
\end{problem}

\bigskip

\begin{problem}
Calculate the partial derivatives and Hessian matrices:
\begin{enumerate}
        \item[a) ] $f(x,y)=3x-2y^2$,
        \item[b) ] $f(x,y)=y^7-2x^3+x^2$,
        \item[c) ] $f(x,y)=\sin{xy}$,
        \item[d) ] $f(x,y)=x^y$,
        \item[e) ] $f(x,y)=\dfrac{x}{y}$.
    \end{enumerate}
\end{problem}

\newpage

\begin{problem}
Calculate the directional derivative:
\begin{enumerate}
        \item[a) ] $f(x, y) = x^2+y^2$ at the point $(1, 2)$ in the direction of $\dfrac{\vv}{\|\vv\|},\,\vv = (3,-4)$,
        \item[b) ] $f(x, y) = \sin(xy)$ at the point $(\pi, 0)$ in the direction of $\dfrac{\vv}{\|\vv\|},\,\vv = (1,1)$.
    \end{enumerate}
\end{problem}

\bigskip

\begin{problem}
Find, if they exist, the local extrema of $f(\vx)=f(x,y)$:
\begin{enumerate}
        \item[a) ] $f(x,y)=x^3-y$,
        \item[b) ] $f(x,y)=x^2-xy$.
    \end{enumerate}
\end{problem}
\bigskip

\begin{problem}
Suppose we roll two fair dice. What is the probability of getting
\begin{enumerate}
    \item[a) ] $2$ on each of them,
    \item[b) ] at least one $1$,
    \item[c) ] exactly one $1$,
    \item[d) ] one $1$ and one $4$,
    \item[e) ] $1$ on the first one and $4$ on the second one?
\end{enumerate}
\end{problem}
\bigskip

\begin{problem}
There are $2$ red, $5$ blue and $6$ yellow pencils in the box. We take two of them out randomly. What is the probability that both of them are
\begin{enumerate}
    \item[a) ] red,
    \item[b) ] of the same color,
    \item[c) ] of different colors,
    \item[d) ] not yellow,
    \item[e) ] not green.
\end{enumerate}
\end{problem}
\bigskip


\begin{problem}
Suppose we roll two fair dice. What is the probability of getting
\begin{enumerate}
    \item[a) ] $5$ on each of them, given that the sum of the resulting numbers is divisible by $5$.
    \item[b) ] at least one $6$, given that the sum of the two numbers is $8$.
\end{enumerate}
\end{problem}

\newpage

\begin{center}
    \begin{large}
        Solutions
    \end{large}
\end{center}


\bigskip

\begin{solution}
\begin{enumerate}
    \item[e) ] $\displaystyle \int \dfrac{2x}{1-x^2} \, dx = \int \dfrac{(x^2)'}{1-x^2} \, dx = \int \dfrac{1}{1-x^2} \, d(x^2) = \int \dfrac{1}{1-y} \, dy=-\ln|1-y|+C=-\ln|1-x^2|+C$,
    \item[f) ] $\displaystyle \int \text{tg}{x} \, dx = \int \dfrac{\sin x}{\cos x} \, dx =  - \int \dfrac{(\cos x)'}{\cos x} \, dx =  - \int \dfrac{1}{\cos x} \, d(\cos x)=  - \int \dfrac{1}{y} \, dy=  - \ln|y|+C=  - \ln|\cos x|+C$.
\end{enumerate}
\end{solution}
\smallskip



\begin{solution}
\begin{enumerate}
        \item[b) ] $f(x,y)=y^7-2x^3+x^2$,
        \begin{align*}
            & f_x(x,y) = -6x^2 + 2x, \\
            & f_y(x,y) = 7y^6, \\
            & H(x,y) = \begin{bmatrix}
                -12x+2 & 0 \\
                0 & 42y^5
            \end{bmatrix}
        \end{align*}
        
        \item[c) ] $f(x,y)=\sin{xy}$,
                \begin{align*}
            & f_x(x,y)=y\cos{xy}, \\
            & f_y(x,y)=x\cos{xy},  \\
            & H(x,y) = \begin{bmatrix}
                -y^2 \sin{xy} & \cos{xy} - xy \sin{xy} \\
                \cos{xy} - xy \sin{xy} & -x^2 \sin{xy}
            \end{bmatrix}
        \end{align*}
        
        \item[d) ] $f(x,y)=x^y$,
                        \begin{align*}
            & f_x(x,y)=yx^{y-1}, \\
            & f_y(x,y)=x^{y}\cdot \ln|x|, \\
            & H(x,y) = \begin{bmatrix}
                y(y-1)x^{y-2} & x^{y-1} + yx^{y-1}\ln|x| \\
                x^{y-1} + yx^{y-1}\ln|x| & x^y (\ln|x|)^2
            \end{bmatrix}
        \end{align*}
\end{enumerate}
\end{solution}
\smallskip


\begin{solution}
\begin{enumerate}
        \item[a) ] First let's denote by a separate variable $\vu = \dfrac{\vv}{\|\vv\|}=\begin{bmatrix}
            3/5\\-4/5
        \end{bmatrix}$. Then we calculate:
        \[ f_x(x,y) = 2x, \qquad f_x(1,2)=2, \]
        \[ f_y(x,y) = 2y, \qquad f_y(1,2)=4, \]
        so the directional derivative of $f(x,y)$ at the given point is:
        \[ \nabla_\vu f(1,2) = f_x(1,2)\cdot u_1+ f_y(1,2)\cdot u_2 = 2\cdot \dfrac{3}{5} + 4 \cdot \left(-\dfrac{4}{5}\right) = -2. \]
        
        \item[b) ] First let's denote by a separate variable $\vu = \dfrac{\vv}{\|\vv\|}=\begin{bmatrix}
            1/\sqrt{2}\\1/\sqrt{2}
        \end{bmatrix}$. Then we calculate:
        \[ f_x(x,y) = y\cos(xy), \qquad f_x(\pi,0)=0, \]
        \[ f_y(x,y) = x\cos(xy), \qquad f_y(\pi,0)=\pi, \]
        so the directional derivative of $f(x,y)$ at the given point is:
        \[ \nabla_\vu f(\pi,0) = f_x(\pi,0)\cdot u_1+ f_y(\pi,0)\cdot u_2 = \dfrac{\pi}{\sqrt{2}}. \]

        
\end{enumerate}
\end{solution}
\smallskip



\begin{solution}
\begin{enumerate}
        \item[a) ] $f(x,y)=x^3-y$
        
        The domain of this function is $(x,y)\in\R^2$. First we need to find its critical points:
\[ f_x(x,y)=3x^2, \]
\[ f_y(x,y)=-1. \]
Since $f_y(x,y)\ne 0$ for any $(x,y)$, this function has no critical points, hence no extremum point.

        
        \item[b) ] $f(x,y)=x^2-xy$
        
    The domain of this function is $(x,y)\in\R^2$. First we need to find its critical points:
\[ f_x(x,y)=2x-y=0,\]
\[ f_y(x,y)=-x=0. \]
The only critical point is $\vx = \begin{bmatrix}
    0 \\ 0
\end{bmatrix}$. In order to find whether it's an extremum or a saddle point, we need to check if its Hessian matrix is positive definite, negative definite or neither of those.
\[  f_{xx}(x,y) = 2,  \]
\[  f_{xy}(x,y)=f_{yx}(x,y) = -1,  \]
\[  f_{xx}(y,y) = 0,  \]
hence its Hessian is:
\[ H(x,y) = H(0,0) = \begin{bmatrix}2&-1\\-1&0\end{bmatrix}. \]
Finally, we check its definiteness:
\[ \Delta_1 = 2 > 0, \]
\[ \Delta_2 = \left|\begin{array}{cc}
    2 & -1 \\
    -1 & 0
\end{array}\right| =-1< 0, \]
therefore the Hessian matrix is neither positive, nor negative definite. It follows that $\vx=\begin{bmatrix}
    0\\0
\end{bmatrix}$ is a saddle point, hence this function has no extremum point.

        
\end{enumerate}
\end{solution}
\smallskip

\begin{solution}
\begin{enumerate}
    \item[a)] Assuming an equiprobable probability space $(\Omega,\F,\PP)$, let's denote the event of getting $2$ on the first die by $A$, and on the second die by $B$. Since $A$ and $B$ are independent,
\[
\PP(A \cap B) = \PP(A)\cdot\PP(B) = \dfrac{1}{6}\cdot\dfrac{1}{6} = \dfrac{1}{36}.
\]

    \item[b)] Since 
    \[
     \PP(\text{getting at least one }1) = 1 - \PP(\text{getting no }1\text{s}),
    \]
    and getting no $1$s means having $2$, $3$, $4$, $5$ or $6$ on both dice,
    \[
      \PP(\text{getting at least one }1) = 1 - \left(\dfrac{5}{6}\right)^2 = \dfrac{11}{36}.
    \]

    An alternative approach would be:
        \[
      \PP(\text{getting at least one }1) =\PP(\text{getting exactly one }1) + \PP(\text{getting two }2\text{s})  = 2\cdot\dfrac{1}{6}\cdot \dfrac{5}{6} + \dfrac{1}{6}\cdot \dfrac{1}{6} = \dfrac{11}{36}.
    \]

    
    \item[c)] $\dfrac{5}{18}$.


    \item[d)] There are two desired outcomes:
    \[
            A = \{(1, 4), (4,1)\},
    \]
    therefore the probability of getting one $1$ and one $4$ is $\dfrac{2}{36} = \dfrac{1}{18}$.
    

    \item[e)] There is only one desired outcome $(1,4)$, hence $\dfrac{1}{36}$.
    
\end{enumerate}



\end{solution}
\smallskip


\begin{solution} There are $13$ pencils in total, out of which we can select $|\Omega| = C^2_{13}=78$ pairs.
\begin{enumerate}
    \item[a)] There is only one possible way to select the only two red pencils ($|A|=C^2_2=1$), so $\PP(\text{both red})=\dfrac{1}{78}$.

    \item[b)] $
     \PP(\text{same color}) = 
        \PP({\text{both red}}) 
        + \PP({\text{both blue}})
        + \PP({\text{both yellow}})
        = \dfrac{C_{2}^2+C_{5}^2+C_{6}^2}{78} = \dfrac{1}{3},
    $
    or, if we denote the event of both pencils being red by $R$, blue by $B$ and yellow by $Y$,
    \[
        \PP(\text{same color}) = \PP(R \cup B \cup Y) = \PP(R) + \PP(B) + \PP(Y) = \dfrac{2}{13} \cdot \dfrac{1}{12}
        + \dfrac{5}{13} \cdot \dfrac{4}{12}
        + \dfrac{6}{13} \cdot \dfrac{5}{12} = \dfrac{1}{3}.
    \]
    
    \item[c)] $\PP(\text{different colors}) = 1 - \PP(\text{same color}) = 1 - \dfrac{1}{3} = \dfrac{2}{3}$.


    \item[d)] Saying that both of them are not yellow is equivalent to say each of them is either red or blue. Out of all possible outcomes we can take $C_{2+5}^2=21$ combinations without any yellow pencils, hence the probability is $\dfrac{21}{78}=\dfrac{7}{26}$.
    

    \item[e)] Zero, as none of them is green.
    
\end{enumerate}



\end{solution}
\smallskip

\begin{solution}
\begin{enumerate}
    \item[a)] Let's denote the event of getting $5$ on each of the dice by $A$, and by $B$, the event of their sum being divisible by $5$. Then,
\begin{align*}
    A &= \{(5,5)\}, \\
    B& = \{(x,y) \mid x+y = 5k\} =  \{(x,y) \mid x+y \in \{5,10\}\} \\&= \{(1, 4), (2, 3), (3, 2), (4, 1), (4, 6), (5, 5), (6,4)\}.
\end{align*}

We have:
\[
\PP(A) = \dfrac{1}{36}, \qquad \PP(B)=\dfrac{7}{36},
\]
and because $A\subset B$,
\[
\PP(A\mid B) = \dfrac{\PP(A\cap B)}{\PP(B)} = \dfrac{\PP(A)}{\PP(B)} = \dfrac{1/36}{7/36} = \dfrac{1}{7}. 
\]

    \item[b)] There are only five outcomes where the sum of the numbers equals $8$, and in only two of them one of the dice is $6$, so we get $\dfrac{2}{5}=0.4$.

    An alternative approach would be denoting the event of having at least one $6$ by $A$, and the sum being $8$ by $B$, whence
\[
 \PP(A \mid B) = \dfrac{\PP(A\cap B)}{\PP(B)} = \dfrac{2/36}{5/36} = 0.4.
\]
\end{enumerate}
\end{solution}
\smallskip
