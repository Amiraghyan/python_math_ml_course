\documentclass[a4paper,12pt]{article}
\usepackage{amsmath}
\usepackage{amsfonts}
\usepackage{amssymb}
\usepackage{geometry}
\usepackage{pgfplots}
\pgfplotsset{compat=1.18}
\geometry{margin=1in}

\title{ACA, Homework 4, Derivatives of Activation Functions (we'll learn what that means later)}
\author{Hayk Aprikyan, Hayk Tarkhanyan}

\begin{document}

\maketitle

Compute the derivatives of the following 3 functions.


\subsection*{Exercise 1: ReLU (Rectified Linear Unit)}
The ReLU function is defined as:
\[
f(x) = \max\{{x, 0}\} = 
\begin{cases} 
x & \text{if } x \geq 0, \\
0 & \text{if } x < 0.
\end{cases}
\]

\subsubsection*{Plot:}
\begin{center}
\begin{tikzpicture}
\begin{axis}[
    axis lines = middle,
    xlabel = {$x$},
    ylabel = {$f(x)$},
    domain = -2:2,
    samples = 100,
    ymin = -1,
    ymax = 2,
    width=0.8\textwidth
]
\addplot[color=blue, thick] {max(0,x)};
\end{axis}
\end{tikzpicture}
\end{center}
\newpage

\subsection*{Exercise 2: Sigmoid Function}
The Sigmoid function is defined as:
\[
f(x) = \frac{1}{1 + e^{-x}}.
\]

\subsubsection*{Plot:}
\begin{center}
\begin{tikzpicture}
\begin{axis}[
    axis lines = middle,
    xlabel = {$x$},
    ylabel = {$f(x)$},
    domain = -6:6,
    samples = 100,
    ymin = 0,
    ymax = 1.1,
    width=0.8\textwidth
]
\addplot[color=blue, thick] {1 / (1 + exp(-x))};
\end{axis}
\end{tikzpicture}
\end{center}

\newpage
\subsection*{Exercise 3: Tanh (Hyperbolic Tangent)}
The Tanh function is defined as:
\[
f(x) = \tanh(x) = \frac{e^x - e^{-x}}{e^x + e^{-x}}.
\]

\subsection*{Plot:}
\begin{center}
\begin{tikzpicture}
\begin{axis}[
    axis lines = middle,
    xlabel = {$x$},
    ylabel = {$f(x)$},
    domain = -4:4,
    samples = 100,
    ymin = -1.1,
    ymax = 1.1,
    width=0.8\textwidth
]
\addplot[color=blue, thick] {tanh(x)};
\end{axis}
\end{tikzpicture}
\end{center}

\end{document}
