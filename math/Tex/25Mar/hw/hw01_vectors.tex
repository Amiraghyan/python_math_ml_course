 \begin{center}\begin{large} Homework Problems 1 (Vectors) \end{large}\end{center}
 \bigskip


\begin{problem}%[1 point]
    The following vectors are given:
\[
\va= \begin{bmatrix} 3 \\ -1 \\ 2 \end{bmatrix}, \qquad \vb=\begin{bmatrix} 2 \\ 4 \\ -5 \end{bmatrix}, \qquad \vc = \begin{bmatrix}
    5 \\ 6.5 \\ -7
\end{bmatrix}
\]

Find the values of:

    \begin{enumerate}
        \item[a) ] $\va+\vb$,
        
        \item[b) ] $\va+\vb-\vc$,

        \item[c) ] $\va^T+\vb^T-\vc^T$,

        \item[d) ] $3\va^T+4\vb^T-2\vc^T$,
        
        \item[e) ] $6\va^T+8\vb^T-4\vc^T$,
        
        \item[f) ] $\va \cdot \vb$,
        
            \item[g) ] $(3\va - 6\vb) \cdot \vc$,
            \item[h) ] $\va \cdot (4\va^T - \vb^T)^T$.
        \end{enumerate}
\end{problem}

% \hspace{20cm}
% \newpage

% Ներքևը կա լուծումը

% \begin{sol}
     
%     % \subsubsection*{Part (c)}
%     % \[
%     % \vu \cdot \vv = \begin{bmatrix} 4 \\ -1 \\ 6 \end{bmatrix} \cdot \begin{bmatrix} 5 \\ 1 \\ 2 \end{bmatrix} = 4 \cdot 5 + (-1) \cdot 1 + 6 \cdot 2
%     % \]
%     % \[
%     % = 20 - 1 + 12 = \textcolor{orange}{31}
%     % \]
    
%     % \subsubsection*{Part (d)}
%     % \[
%     % (3\vv - 6\vu) \cdot \vv = \left(3 \cdot \begin{bmatrix} 5 \\ 1 \\ 2 \end{bmatrix} - 6 \cdot \begin{bmatrix} 4 \\ -1 \\ 6 \end{bmatrix}\right) \cdot \begin{bmatrix} 5 \\ 1 \\ 2 \end{bmatrix}
%     % \]
%     % \[
%     % = \left(\begin{bmatrix} 15 \\ 3 \\ 6 \end{bmatrix} - \begin{bmatrix} 24 \\ -6 \\ 36 \end{bmatrix}\right) \cdot \begin{bmatrix} 5 \\ 1 \\ 2 \end{bmatrix}
%     % \]
%     % \[
%     % = \begin{bmatrix} -9 \\ 9 \\ -30 \end{bmatrix} \cdot \begin{bmatrix} 5 \\ 1 \\ 2 \end{bmatrix} = -9 \cdot 5 + 9 \cdot 1 + (-30) \cdot 2
%     % \]
%     % \[
%     % = -45 + 9 - 60 = \textcolor{orange}{-96}
%     % \]
    
%     % \subsubsection*{Part (e)}
%     % \[
%     % \vv \cdot (4\vv^T - \vu^T)^T = \vv \cdot (4 \cdot \vv - \vu)
%     % \]
%     % \[
%     % = \begin{bmatrix} 5 \\ 1 \\ 2 \end{bmatrix} \cdot \left(4 \cdot \begin{bmatrix} 5 \\ 1 \\ 2 \end{bmatrix} - \begin{bmatrix} 4 \\ -1 \\ 6 \end{bmatrix}\right)
%     % \]
%     % \[
%     % = \begin{bmatrix} 5 \\ 1 \\ 2 \end{bmatrix} \cdot \left(\begin{bmatrix} 20 \\ 4 \\ 8 \end{bmatrix} - \begin{bmatrix} 4 \\ -1 \\ 6 \end{bmatrix}\right)
%     % \]
%     % \[
%     % = \begin{bmatrix} 5 \\ 1 \\ 2 \end{bmatrix} \cdot \begin{bmatrix} 16 \\ 5 \\ 2 \end{bmatrix} = 5 \cdot 16 + 1 \cdot 5 + 2 \cdot 2
%     % \]
%     % \[
%     % = 80 + 5 + 4 = \textcolor{orange}{89}
%     % \]
% \end{sol}


\begin{problem}%[1 point]
    A translation office translated $\va=[24, 17, 9, 13]$ documents from English, French, German and Russian, respectively. For each of those languages, it takes about $\vb=[5, 10, 11, 7]$ minutes to translate one page.

\smallskip
    
    How much time did they spend translating in total? How much did each of the translators spend on average if there are $4$ translators in the office? Write an expression for this amount in terms of the vectors $\va$ and $\vb$.
\end{problem}



% \begin{sol}
%     \subsubsection*{Total Time Spent}

%     The total time spent translating is the dot product of the vectors \(\va\) and \(\vb\):
%     \[
%     \text{Total Time} = \va \cdot \vb = \begin{bmatrix} 24 \\ 17 \\ 9 \\ 13 \end{bmatrix} \cdot \begin{bmatrix} 5 \\ 10 \\ 11 \\ 7 \end{bmatrix}
%     \]
%     \[
%     = (24 \cdot 5) + (17 \cdot 10) + (9 \cdot 11) + (13 \cdot 7)
%     \]
%     \[
%     = 120 + 170 + 99 + 91 = \textcolor{orange}{480 \text{ minutes}}.
%     \]
    
%     \subsubsection*{Average Time per Translator}
    
%     The average time per translator, given there are \(4\) translators, is:
%     \[
%     \text{Average Time} = \frac{\text{Total Time}}{4} = \frac{480}{4} = \textcolor{orange}{120 \text{ minutes}}.
%     \]
    
%     \subsubsection*{Expression in Terms of \(\va\) and \(\vb\)}
    
%     The total time in terms of \(\va\) and \(\vb\) is:
%     \[
%     \text{Total Time} = \va \cdot \vb
%     \]
    
%     The average time per translator is:
%     \[
%     \text{Average Time} = \frac{\va \cdot \vb}{4}
%     \]
% \end{sol}


\medskip

\begin{problem}
    Calculate the Manhattan (L1) and Euclidean (L2) norms of the following vectors:

    \begin{enumerate}
        \item[a) ] $\va= \begin{bmatrix} 2\\-9\\3 \end{bmatrix}$,
        
        \item[b) ] $\va-2\vb$, where $\va= \begin{bmatrix}3\\4\\1\\0\end{bmatrix},\  \vb = \begin{bmatrix}4\\5\\-2\\-1\end{bmatrix}$,
        
        \item[c) ] $-3\vc$, where $\vc= \begin{bmatrix}2\\-5\\6\end{bmatrix}$.
    \end{enumerate}
    % \bigskip

    %     \textbf{Optional, not graded}: Find the angle between $\va$ in \textit{a)} and $\vc$ in \textit{c)}.
\end{problem}
\medskip



\begin{problem}
    Find the angles between the following vectors:

    \begin{enumerate}
        \item[a) ] $\va= \begin{bmatrix} 2\\1\\1\end{bmatrix}$ and $\vb= \begin{bmatrix} 1\\-3\\3\end{bmatrix}$,
        
        \item[b) ] The vectors $\va$ and $\vc$ in the previous problem.

    \end{enumerate}
\end{problem}
\medskip




% \begin{problem}[1 point]
%     Given the following system of equations, find values for $a$ and $b$:
%    \[ \begin{cases}
%          a + 3b = 4 \\
%          a - b = -4 
%     \end{cases}\]

%     How can we write the equations in terms of matrices?
% \end{problem}



\begin{problem}
    What vectors do you get by applying the matrix
    \[
    A = \begin{bmatrix}
        3 & -3 \\
        3 & 3
    \end{bmatrix}
    \] 
    on the vectors 
    \begin{enumerate}
        \item[a) ] $\va=\begin{bmatrix}
            1 \\ 0
        \end{bmatrix}$,

          \item[b) ] $\vb=\begin{bmatrix}
            0 \\ 1
        \end{bmatrix}$,

          \item[c) ] $\vc=\begin{bmatrix}
            1 \\ 1
        \end{bmatrix}$?
    \end{enumerate}
    
    (Additional:) Draw the vectors before and after multiplying with $A$. What can you say visually about the matrix? Can you guess how it will act on the vector $[2\,\,\,\,\,\,\,\, -2]$?
    \end{problem}

    \medskip
    \begin{problem}
        
Compute the following products:

    \begin{enumerate}
        \item[a) ] $AB$, where $A=\begin{bmatrix}
        6&5\\-2&7  \end{bmatrix}$, $B=\begin{bmatrix}
        -5&3\\1&4   \end{bmatrix}$,

        \item[b) ]  $(A-B)(A+B)$, where $A=\begin{bmatrix}
        2&2&4\\-3&-2&4\\-2&0&2   \end{bmatrix}$, $B=\begin{bmatrix}
        2&1&3\\-1&2&2\\1&4&-1   \end{bmatrix}$,
        
        \item[c) ] $A^2 - B^2$, with the same $A$ and $B$ as in \textit{b)}.
    \end{enumerate}
\end{problem}

\medskip




\begin{problem}[\textbf{additional}]%[2 points]
    Consider the following matrix (it is called the \textit{shear matrix}):
    \[
    S = \begin{bmatrix}
        1 & 1 \\ 0 & 1
    \end{bmatrix}
    \]
    \begin{enumerate}
        \item[a) ] What would you get if you apply $S$ on the vector $[0 \qquad 1]$?
        
        \item[b) ] What would you get if you apply $S$ again on the result of the previous point?

        \item[c) ] What if you apply $S$ one more time?

        \item[d) ] What do you think happens when we apply $S$ 100 times on that vector?

        \item[e) ] Can you compute $S^{100}$?
        
        
    \end{enumerate}

\end{problem}

\medskip

\begin{problem}[\textbf{additional}]%[2 points]
    Show that the following set is a vector space:

    \begin{enumerate}
        \item [a)] $A = \bigg\{ \begin{bmatrix} a \\ 0 \end{bmatrix} \mid \text{ for all numbers }a\in\R\bigg\}$,
        
        \item [b)] $B = \bigg\{ \begin{bmatrix} a \\ -a \end{bmatrix} \mid \text{ for all numbers }a\in\R\bigg\}$.

        % \item [c)] The set of all polynomials of the form $ax+b$ \\(i.e. all polynomials of degree $\le 1$). 

    \end{enumerate}
    
\end{problem}
\medskip
\begin{problem}[\textbf{additional}]%[2 points]
    Show that the following set is \textit{not} a vector space:

    \begin{enumerate}
        \item [a)] $A = \N$,
        
        \item [b)] $B =  \bigg\{ \begin{bmatrix} a \\ 1 \end{bmatrix} \mid \text{ for all numbers }a\in\R\bigg\}$.
        
        % \item [c)] The set of all polynomials of the form $ax+b$ where $a \ne 0$.

    \end{enumerate}
    \bigskip
    {\small \textit{Hint:} \textit{Show that there are some "bad" elements, such that if we add them or multiply with some number (not necessarily positive), the result would not belong to the set.}}
\end{problem}


        
        