\begin{center}\begin{large} Homework 2 Solutions
 \end{large}\end{center}
 \bigskip

% \documentclass{article}
% \usepackage{amsmath}

% \begin{document}


\begin{problem}[3 points]
Check if the following set is a vector space:

    \begin{enumerate}
        \item [a)] The set of real negative numbers $A = \{x\in\R \mid x < 0\}$, with the usual operations $+$ and $\cdot$,

        \item [b)] $B = \left\{ \begin{bmatrix} a \\ b \\ c\end{bmatrix} \mid \text{ for all real numbers }a, b, c\in\R\right\}$, with the usual operation of $\cdot$, and the addition defined as:
        \[ \begin{bmatrix} a_1 \\ b_1 \\ c_1\end{bmatrix} + \begin{bmatrix} a_2 \\ b_2 \\ c_2\end{bmatrix} = \begin{bmatrix} a_1 + a_2 \\ b_1 + b_2\\ c_1 + c_2 - 1\end{bmatrix} ,\]
        
        \item [c)] $C =  \bigg\{ \begin{bmatrix} a \\ b \end{bmatrix} \mid \text{ for all numbers }a,b\in\R\bigg\}$, with the usual operation of $\cdot$, and the addition defined as:
        \[ \begin{bmatrix} a_1 \\ b_1 \end{bmatrix} +\begin{bmatrix} a_2 \\ b_2 \end{bmatrix} = \begin{bmatrix} a_1-a_2 \\ b_1-b_2 \end{bmatrix} , \]

        \item [d)] $D =  \bigg\{ \begin{bmatrix} a \\ b \end{bmatrix} \mid \text{ for all numbers }a,b\in\R\bigg\}$, with the usual operation of $\cdot$, and the addition defined as:
        \[ \begin{bmatrix} a_1 \\ b_1 \end{bmatrix} +\begin{bmatrix} a_2 \\ b_2 \end{bmatrix} = \begin{bmatrix} 0 \\ 0\end{bmatrix}  ,\]
        
        \item [e)] $E =  \bigg\{ \begin{bmatrix} a \\ b \end{bmatrix} \mid \text{ for all numbers }a,b\in\R\bigg\}$, with the usual operation of $+$, and the scalar multiplication defined as:
        \[ c \cdot \begin{bmatrix} a \\ b \end{bmatrix} = \begin{bmatrix} c \cdot ( a+b) \\ cb\end{bmatrix}  , \qquad \text{for any } c>0.\]

        \item [f)] The set of all polynomials of degree $2$, with the usual operations $+$ and $\cdot$.

        
    \end{enumerate}
\end{problem}

\begin{sol}
    \begin{enumerate}
    {
        \item [a)]
    To determine if the set \( A = \{ x \in \mathbb{R} \mid x < 0 \} \) is a vector space, we need to verify if it satisfies all the vector space axioms under the usual operations of addition \( + \) and scalar multiplication \( \cdot \).

    First of all, the following 2 properties should hold:
    \begin{itemize}
        \item \textbf{Closure under addition}: For any \( x, y \in A \), \( x + y \) should also be in \( A \).
        
        \item \textbf{Closure under scalar multiplication}: For any \( x \in A \) and any scalar \( c \in \mathbb{R} \), \( c \cdot x \) should also be in \( A \).
    \end{itemize}

    Let's check.

    \begin{enumerate}
        \item \textbf{Closure under addition}: Let \( x, y \in A \), meaning \( x < 0 \) and \( y < 0 \). Then \( x + y < 0 \) (since the sum of two negative numbers is also negative). Therefore, \( x + y \in A \), so \( A \) is closed under addition.

        \item \textbf{Closure under scalar multiplication}: For \( x \in A \) and any scalar \( c \in \mathbb{R} \):
        
        - If \( c > 0 \), then \( c \cdot x < 0 \) since \( x < 0 \). \\
        - If \( c < 0 \), then \( c \cdot x > 0 \), which is not in \( A \). \\
        - If \( c = 0 \), then \( c \cdot x = 0 \), which is also not in \( A \) (since \( 0 \not< 0 \)).

        Thus, \( A \) is not closed under scalar multiplication.
    \end{enumerate}
    No need to do anything else, showing that something is \textbf{not} a vector space is much more pleasant because we can just stop after one axiom breaks. On the other hand, if it is a vector space than we must check every axiom.
    }

    \item[b)] ToDo
    \item[c)] ToDo
    \item[d)] ToDo
    \item[e)] ToDo
    \item[f)] ToDo
    
    \end{enumerate}
    
\end{sol}

\newpage

\begin{problem}[2 point]
    Calculate the Manhattan (L1) and Euclidean (L2) norms of the following vectors:

    \begin{enumerate}
        \item[a) ] $\va= \begin{bmatrix} 2\\-9\\3 \end{bmatrix}$,
        
        \item[b) ] $\va-2\vb$, where $\va= \begin{bmatrix}3\\4\\1\\0\end{bmatrix}, \vb = \begin{bmatrix}4\\5\\-2\\-1\end{bmatrix}$,
        
        \item[c) ] $-3\vc$, where $\vc= \begin{bmatrix}2\\-5\\6\end{bmatrix}$.
    \end{enumerate}
    % \bigskip

    %     \textbf{Optional, not graded}: Find the angle between $\va$ in \textit{a)} and $\vc$ in \textit{c)}.
\end{problem}

% 2
\begin{sol}
   Let's first recall the definitions:

    For a vector $\vx = \begin{bmatrix} x_1 & x_2 & \dots & x_n \end{bmatrix}^\top$

    - \textbf{Manhattan (L1) norm}: 
      \[
      \|\vx\|_1 = |x_1| + |x_2| + \dots + |x_n|
      \]
      
    - \textbf{Euclidean (L2) norm}: 
      \[
      \|\vx\|_2 = \sqrt{x_1^2 + x_2^2 + \dots + x_n^2}
      \]
      
    Now let's get to work:

    \begin{enumerate}
        \item[a)] For $\va= \begin{bmatrix} 2 \\ -9 \\ 3 \end{bmatrix}$:
        
        \[
        \|\va\|_1 = |2| + |-9| + |3| = 2 + 9 + 3 = 14
        \]
        
        \[
        \|\va\|_2 = \sqrt{2^2 + (-9)^2 + 3^2} = \sqrt{4 + 81 + 9} = \sqrt{94}
        \]
        
        \item[b)] For $\va - 2\vb$, where $\va= \begin{bmatrix} 3 \\ 4 \\ 1 \\ 0 \end{bmatrix}$ and $\vb = \begin{bmatrix} 4 \\ 5 \\ -2 \\ -1 \end{bmatrix}$:
        
        First, compute $\va - 2\vb$:
        
        \[
        \va - 2\vb = \begin{bmatrix} 3 \\ 4 \\ 1 \\ 0 \end{bmatrix} - 2 \cdot \begin{bmatrix} 4 \\ 5 \\ -2 \\ -1 \end{bmatrix} = \begin{bmatrix} 3 \\ 4 \\ 1 \\ 0 \end{bmatrix} - \begin{bmatrix} 8 \\ 10 \\ -4 \\ -2 \end{bmatrix} = \begin{bmatrix} -5 \\ -6 \\ 5 \\ 2 \end{bmatrix}
        \]
        
        Now calculate the norms:
        
        \[
        \|\va - 2\vb\|_1 = |-5| + |-6| + |5| + |2| = 5 + 6 + 5 + 2 = 18
        \]
        
        \[
        \|\va - 2\vb\|_2 = \sqrt{(-5)^2 + (-6)^2 + 5^2 + 2^2} = \sqrt{25 + 36 + 25 + 4} = \sqrt{90}
        \]
        
        \item[c)] For $-3\vc$, where $\vc= \begin{bmatrix} 2 \\ -5 \\ 6 \end{bmatrix}$:
        
        First, compute $-3\vc$:
        
        \[
        -3\vc = -3 \cdot \begin{bmatrix} 2 \\ -5 \\ 6 \end{bmatrix} = \begin{bmatrix} -6 \\ 15 \\ -18 \end{bmatrix}
        \]
        
        Now calculate the norms:
        
        \[
        \|-3\vc\|_1 = |-6| + |15| + |-18| = 6 + 15 + 18 = 39
        \]
        
        \[
        \|-3\vc\|_2 = \sqrt{(-6)^2 + 15^2 + (-18)^2} = \sqrt{36 + 225 + 324} = \sqrt{585}
        \]
    \end{enumerate}

    Thus, the solutions are:
    
    \begin{enumerate}
        \item[a)] $\|\va\|_1 = 14$, \quad $\|\va\|_2 = \sqrt{94}$
        \item[b)] $\|\va - 2\vb\|_1 = 18$, \quad $\|\va - 2\vb\|_2 = \sqrt{90}$
        \item[c)] $\|-3\vc\|_1 = 39$, \quad $\|-3\vc\|_2 = \sqrt{585}$
    \end{enumerate}
\end{sol}

\newpage

% 3
\begin{problem}[1 point]
    Find the angles between the following vectors:

    \begin{enumerate}
        \item[a) ] $\va= \begin{bmatrix} 2\\1\\1\end{bmatrix}$ and $\vb= \begin{bmatrix} 1\\-3\\3\end{bmatrix}$,
        
        \item[b) ] The vectors $\va$ and $\vc$ in Problem 2a and 2c.

    \end{enumerate}
\end{problem}
\bigskip

\begin{sol}
    To find the angle $\theta$ between two vectors $\va$ and $\vb$, we use the following formula:

    \[
    \cos \theta = \frac{\va \cdot \vb}{\|\va\| \|\vb\|}
    \]

    where $\va \cdot \vb$ is the dot product of $\va$ and $\vb$, and $\|\va\|$ and $\|\vb\|$ are the magnitudes (lengths) (Euclidean norms) of $\va$ and $\vb$.

    \begin{enumerate}
        \item[a)] For $\va= \begin{bmatrix} 2 \\ 1 \\ 1 \end{bmatrix}$ and $\vb= \begin{bmatrix} 1 \\ -3 \\ 3 \end{bmatrix}$:

        - First, compute the dot product $\va \cdot \vb$:
          \[
          \va \cdot \vb = 2 \cdot 1 + 1 \cdot (-3) + 1 \cdot 3 = 2 - 3 + 3 = 2
          \]

        - Next, compute the magnitudes (lengths) $\|\va\|$ and $\|\vb\|$:
          \[
          \|\va\| = \sqrt{2^2 + 1^2 + 1^2} = \sqrt{4 + 1 + 1} = \sqrt{6}
          \]
          \[
          \|\vb\| = \sqrt{1^2 + (-3)^2 + 3^2} = \sqrt{1 + 9 + 9} = \sqrt{19}
          \]

        - Now we can find $\cos \theta$:
          \[
          \cos \theta = \frac{\va \cdot \vb}{\|\va\| \|\vb\|} = \frac{2}{\sqrt{6} \cdot \sqrt{19}} = \frac{2}{\sqrt{114}}
          \]

        - Therefore, the angle $\theta$ between $\va$ and $\vb$ is:
          \[
          \theta = \cos^{-1} \left( \frac{2}{\sqrt{114}} \right)
          \]

        - We don't really care about the actual value of such an unpleasant number, but if you're curious, the answer is approximately 79.2 degrees.

        \item[b)] For vectors $\va$ and $\vc$ from Problem 2a and 2c, where $\va= \begin{bmatrix} 2 \\ -9 \\ 3 \end{bmatrix}$ and $\vc = \begin{bmatrix} 2 \\ -5 \\ 6 \end{bmatrix}$:

        - First, compute the dot product $\va \cdot \vc$:
          \[
          \va \cdot \vc = 2 \cdot 2 + (-9) \cdot (-5) + 3 \cdot 6 = 4 + 45 + 18 = 67
          \]

        - Next, compute the magnitudes $\|\va\|$ and $\|\vc\|$:
          \[
          \|\va\| = \sqrt{2^2 + (-9)^2 + 3^2} = \sqrt{4 + 81 + 9} = \sqrt{94}
          \]
          \[
          \|\vc\| = \sqrt{2^2 + (-5)^2 + 6^2} = \sqrt{4 + 25 + 36} = \sqrt{65}
          \]

        - Now we can find $\cos \theta$:
          \[
          \cos \theta = \frac{\va \cdot \vc}{\|\va\| \|\vc\|} = \frac{67}{\sqrt{94} \cdot \sqrt{65}} = \frac{67}{\sqrt{6110}}
          \]

        - Therefore, the angle $\theta$ between $\va$ and $\vc$ is:
          \[
          \theta = \cos^{-1} \left( \frac{67}{\sqrt{6110}} \right)
          \] 
    \end{enumerate}
\end{sol}



% \begin{problem}[1 point]
%     Given the following system of equations, find values for $a$ and $b$:
%    \[ \begin{cases}
%          a + 3b = 4 \\
%          a - b = -4 
%     \end{cases}\]

%     How can we write the equations in terms of matrices?
% \end{problem}


\newpage
% 4
\begin{problem}[2 points]
    Evaluate the expression:

    \begin{enumerate}
        \item[a) ] $AB$, where $A=\begin{bmatrix}
        6&5\\-2&7  \end{bmatrix}$, $B=\begin{bmatrix}
        -5&3\\1&4   \end{bmatrix}$,

        \item[b) ] $B^3$, where $B=\begin{bmatrix}
        -2&1&4\\1&2&1\\-2&-2&0   \end{bmatrix}$,

        \item[c) ] $CD$, where $C=\begin{bmatrix}
        7&2&-3\\0&2&8\\8&1&-3\\4&3&-1   \end{bmatrix}$, $D=\begin{bmatrix}
            -2&4\\8&1\\-1&5
        \end{bmatrix}$,
        
        \item[d) ] $(A-B)(A+B)$, where $A=\begin{bmatrix}
        2&2&4\\-3&-2&4\\-2&0&2   \end{bmatrix}$, $B=\begin{bmatrix}
        2&1&3\\-1&2&2\\1&4&-1   \end{bmatrix}$,
        
        \item[e) ] $A^2 - B^2$, with the same $A$ and $B$ as in \textit{d)}.
    \end{enumerate}
\end{problem}

\begin{sol}
    This is a super boring exercise, but everyone needs to go through this suffering at least once to get familiar with the pain of multiplying matrices. That being said, let's have some fun:
    \begin{enumerate}
        \item[a)] Compute $AB$, where 
        \[
        A = \begin{bmatrix} 6 & 5 \\ -2 & 7 \end{bmatrix}, \quad B = \begin{bmatrix} -5 & 3 \\ 1 & 4 \end{bmatrix}
        \]

          \[
          AB = \begin{bmatrix} 6 \cdot (-5) + 5 \cdot 1 & 6 \cdot 3 + 5 \cdot 4 \\ -2 \cdot (-5) + 7 \cdot 1 & -2 \cdot 3 + 7 \cdot 4 \end{bmatrix}
          \]
          \[
          = \begin{bmatrix} -30 + 5 & 18 + 20 \\ 10 + 7 & -6 + 28 \end{bmatrix} = \begin{bmatrix} -25 & 38 \\ 17 & 22 \end{bmatrix}
          \]

        \item[b)] Compute $B^3$, where 
        \[
        B = \begin{bmatrix} -2 & 1 & 4 \\ 1 & 2 & 1 \\ -2 & -2 & 0 \end{bmatrix}
        \]

        - First, find $B^2 = B \cdot B$ and then multiply the result by $B$ to get $B^3$.
        
        \[
        B^2 = \begin{bmatrix} -2 & 1 & 4 \\ 1 & 2 & 1 \\ -2 & -2 & 0 \end{bmatrix} \cdot \begin{bmatrix} -2 & 1 & 4 \\ 1 & 2 & 1 \\ -2 & -2 & 0 \end{bmatrix}
        \]
        
        - Calculating each entry of $B^2$ yields:
          \[
          B^2 = \begin{bmatrix} -3 & -8 & -7 \\ -2 & 3 & 6 \\ 2 & -6 & -10 \end{bmatrix}
          \]

        - Now, calculate $B^3 = B^2 \cdot B$:
          \[
          B^3 = \begin{bmatrix} -3 & -8 & -7 \\ -2 & 3 & 6 \\ 2 & -6 & -10 \end{bmatrix} \cdot \begin{bmatrix} -2 & 1 & 4 \\ 1 & 2 & 1 \\ -2 & -2 & 0 \end{bmatrix}
          \]
          
          - Calculating each entry of $B^3$ gives:
            \[
            B^3 = \begin{bmatrix} 12 & -5 & -20 \\ -5 & -8 & -5 \\ 10 & 10 & 2 \end{bmatrix}
            \]

        \item[c)] Compute $CD$, where 
        \[
        C = \begin{bmatrix} 7 & 2 & -3 \\ 0 & 2 & 8 \\ 8 & 1 & -3 \\ 4 & 3 & -1 \end{bmatrix}, \quad D = \begin{bmatrix} -2 & 4 \\ 8 & 1 \\ -1 & 5 \end{bmatrix}
        \]
        - Let's note that we can actually perform matrix multiplication because $C$ has dimensions 4x3 and $D$ has dimensions 3x2. If the number of columns of the first matrix matches the number of rows of the second one than we're good to go. We can also note that the result will be a 4 by 2 matrix.  
        
        - Calculate each entry of $CD$:
          \[
          CD = \begin{bmatrix} 7 \cdot (-2) + 2 \cdot 8 + (-3) \cdot (-1) & 7 \cdot 4 + 2 \cdot 1 + (-3) \cdot 5 \\ 0 \cdot (-2) + 2 \cdot 8 + 8 \cdot (-1) & 0 \cdot 4 + 2 \cdot 1 + 8 \cdot 5 \\ 8 \cdot (-2) + 1 \cdot 8 + (-3) \cdot (-1) & 8 \cdot 4 + 1 \cdot 1 + (-3) \cdot 5 \\ 4 \cdot (-2) + 3 \cdot 8 + (-1) \cdot (-1) & 4 \cdot 4 + 3 \cdot 1 + (-1) \cdot 5 \end{bmatrix}
          \]
          
          \[
          = \begin{bmatrix} 5 & 15 \\ 8 & 42 \\ -5 & 18 \\ 17 & 14 \end{bmatrix}
          \]

        \item[d)] Compute $(A - B)(A + B)$, where 
        \[
        A = \begin{bmatrix} 2 & 2 & 4 \\ -3 & -2 & 4 \\ -2 & 0 & 2 \end{bmatrix}, \quad B = \begin{bmatrix} 2 & 1 & 3 \\ -1 & 2 & 2 \\ 1 & 4 & -1 \end{bmatrix}
        \]

        - First, calculate $A - B$ and $A + B$:
          \[
          A - B = \begin{bmatrix} 2 - 2 & 2 - 1 & 4 - 3 \\ -3 - (-1) & -2 - 2 & 4 - 2 \\ -2 - 1 & 0 - 4 & 2 - (-1) \end{bmatrix} = \begin{bmatrix} 0 & 1 & 1 \\ -2 & -4 & 2 \\ -3 & -4 & 3 \end{bmatrix}
          \]
          \[
          A + B = \begin{bmatrix} 2 + 2 & 2 + 1 & 4 + 3 \\ -3 + (-1) & -2 + 2 & 4 + 2 \\ -2 + 1 & 0 + 4 & 2 + (-1) \end{bmatrix} = \begin{bmatrix} 4 & 3 & 7 \\ -4 & 0 & 6 \\ -1 & 4 & 1 \end{bmatrix}
          \]

        - Now calculate $(A - B)(A + B)$:
          \[
          (A - B)(A + B) = \begin{bmatrix} 0 & 1 & 1 \\ -2 & -4 & 2 \\ -3 & -4 & 3 \end{bmatrix} \cdot \begin{bmatrix} 4 & 3 & 7 \\ -4 & 0 & 6 \\ -1 & 4 & 1 \end{bmatrix} 
          = \begin{bmatrix} -5 & 4 & 7 \\ 6 & 2 & -36 \\ 1 & 3 & -42 \end{bmatrix}
          \]
          
          
        \item[e)] Compute $A^2 - B^2$ for the same $A$ and $B$ as in part (d): \\
        \textbf{\textcolor{red}{WAIT!!!!!}}
        Before starting to do this, do you think you can guess what the answer is? Think for a little while, maybe grab a cup of tea. Ready? Okay, let's go.

        - First, calculate $A^2$ and $B^2$ separately:
          \[
          A^2 = A \cdot A = \begin{bmatrix} 2 & 2 & 4 \\ -3 & -2 & 4 \\ -2 & 0 & 2 \end{bmatrix} \cdot \begin{bmatrix} 2 & 2 & 4 \\ -3 & -2 & 4 \\ -2 & 0 & 2 \end{bmatrix} = \begin{bmatrix} -10 & 0 & 24 \\ -8 & -2 & -12 \\ -8 & -4 & -4 \end{bmatrix}
          \]
          \[
          B^2 = B \cdot B = \begin{bmatrix} 2 & 1 & 3 \\ -1 & 2 & 2 \\ 1 & 4 & -1 \end{bmatrix} \cdot \begin{bmatrix} 2 & 1 & 3 \\ -1 & 2 & 2 \\ 1 & 4 & -1 \end{bmatrix} = \begin{bmatrix} 6 & 16 & 5 \\ -2 & 11 & -1 \\ -3 & 5 & 12 \end{bmatrix}
          \]

        - Now calculate $A^2 - B^2$:
          \[
          A^2 - B^2 = \begin{bmatrix} -10 & 0 & 24 \\ -8 & -2 & -12 \\ -8 & -4 & -4 \end{bmatrix} - \begin{bmatrix} 6 & 16 & 5 \\ -2 & 11 & -1 \\ -3 & 5 & 12 \end{bmatrix} = \begin{bmatrix} -16 & -16 & 19 \\ -6 & -13 & -11 \\ -5 & -9 & -16 \end{bmatrix} 
          \]

        So, the answer is not the same as in the previous exercise. Did you guess correctly? Can you figure out what condition should hold for $A$ and $B$ in order for the answer to be the same. Here (https://math.stackexchange.com/questions/3518069/conditions-for-a2-b2-aba-b-to-be-true) is a link that you can refer to, but only after struggling on your own. We want know if you click the link right away, but we hope you won't do it :)
    \end{enumerate}
\end{sol}


\newpage

% 5
\begin{problem}[2 points]
    Evaluate the expression:

    \begin{enumerate}
        \item[a) ] $\det(A)$, where $A=\begin{bmatrix}
        8&3&5\\1&4&2\\-4&0&4   \end{bmatrix}$,

        \item[b) ] $\text{det}(B)-\text{tr}(B)$, where $B=\begin{bmatrix}
        5&2&1\\4&-1&4\\-3&1&2 \end{bmatrix}$,
        
        \item[c) ] $\text{det}(C)$, where $C=\begin{bmatrix}
        3&0&1&3\\2&0&4&1\\0&2&-1&3\\5&0&0&3   \end{bmatrix}$,
        
        \item[d) ] $\text{det}(D)$, where $D=\begin{bmatrix}
        5&2&-3&2\\1&0&4&1\\0&5&0&-2\\1&1&5&-1  \end{bmatrix}$.
    \end{enumerate}
\end{problem}

\begin{sol}
    This is not the most fun exercise as well, but practice is important.
    \begin{enumerate}
        \item[a)] Compute $\det(A)$, where 
        \[
        A = \begin{bmatrix} 8 & 3 & 5 \\ 1 & 4 & 2 \\ -4 & 0 & 4 \end{bmatrix}
        \]

        Using the formula for a 3x3 determinant with six terms (three positive and three negative), we have:
        \[
        \det(A) = 8 \cdot 4 \cdot 4 + 3 \cdot 2 \cdot (-4) + 5 \cdot 1 \cdot 0 - (5 \cdot 4 \cdot (-4) + 3 \cdot 1 \cdot 4 + 8 \cdot 2 \cdot 0)
        \]
        \[
        = 8 \cdot 16 + 3 \cdot (-8) + 0 - (5 \cdot (-16) + 3 \cdot 4 + 0)
        \]
        \[
        = 128 - 24 + 0 - (-80 + 12 + 0) = 128 - 24 + 80 - 12 = 172
        \]

        So, $\det(A) = 172$.

        \item[b)] Compute $\det(B) - \text{tr}(B)$, where 
        \[
        B = \begin{bmatrix} 5 & 2 & 1 \\ 4 & -1 & 4 \\ -3 & 1 & 2 \end{bmatrix}
        \]

        For $\det(B)$, we use the diagonal method by extending the first two columns to the right:

        \[
        \begin{bmatrix} 
            5 & 2 & 1 & 5 & 2 \\ 
            4 & -1 & 4 & 4 & -1 \\ 
            -3 & 1 & 2 & -3 & 1 
        \end{bmatrix}
        \]

        Now, we apply the formula for the diagonal method:
        \[
        \det(B) = 5 \cdot (-1) \cdot 2 + 2 \cdot 4 \cdot (-3) + 1 \cdot 4 \cdot 1 - (1 \cdot (-1) \cdot (-3) + 5 \cdot 4 \cdot 1 + 2 \cdot 4 \cdot 2)
        \]
        \[
        = -10 - 24 + 4 - (3 + 20 + 16) = -69
        \]

        Therefore, $\det(B) = -69$.


        Now, calculate $\text{tr}(B)$ (the trace of $B$), which is the sum of the diagonal elements:
        \[
        \text{tr}(B) = 5 + (-1) + 2 = 6
        \]

        Therefore:
        \[
        \det(B) - \text{tr}(B) = -69 - 6 = -75
        \]
    
            \item[c)] Compute $\det(C)$, where 
        \[
        C = \begin{bmatrix} 3 & 0 & 1 & 3 \\ 2 & 0 & 4 & 1 \\ 0 & 2 & -1 & 3 \\ 5 & 0 & 0 & 3 \end{bmatrix}
        \]

        Since the second column contains many zeros (and I'm lazy (and you should be as well :)), we expand along this column to simplify the calculation:
        \[
        \det(C) = (-1)^{1+2} \cdot 0 \cdot \begin{vmatrix} 2 & 4 & 1 \\ 0 & -1 & 3 \\ 5 & 0 & 3 \end{vmatrix} + \\ (-1)^{2+2} \cdot 2 \cdot \begin{vmatrix} 3 & 1 & 3 \\ 2 & 4 & 1 \\ 5 & 0 & 3 \end{vmatrix} \\ + (-1)^{3+2} \cdot 0 \cdot \text{doesn't matter} + \\ (-1)^{4+2} \cdot 0
        \cdot \text{...}
        \]

        We only need to compute the second term:
        \[
    = 2 \cdot \begin{vmatrix} 3 & 1 & 3 \\ 2 & 4 & 1 \\ 5 & 0 & 3 \end{vmatrix}
        \]

        Expanding along the first row of this $3 \times 3$ matrix:
        \[
        = 2 \cdot \left( 3 \cdot \begin{vmatrix} 4 & 1 \\ 0 & 3 \end{vmatrix} - 1 \cdot \begin{vmatrix} 2 & 1 \\ 5 & 3 \end{vmatrix} + 3 \cdot \begin{vmatrix} 2 & 4 \\ 5 & 0 \end{vmatrix} \right)
        \]
        And now more boring calculations:
        \[
        = 2 \cdot \left( 3 \cdot (4 \cdot 3 - 1 \cdot 0) - 1 \cdot (2 \cdot 3 - 1 \cdot 5) + 3 \cdot (2 \cdot 0 - 4 \cdot 5) \right)
        \]
        \[
        = 2 \cdot \left( 3 \cdot 12 - 1 \cdot 1 - 3 \cdot -20 \right) = 2 \cdot (36 + 5 + 60) = 2 \cdot 50
        \]

        Therefore, $\det(C) = 50$.

        \textbf{Note:}
        If you just asked "Why didn't we expand along the third row?" - Good job, go grab a chocolate, you deserve it.

        \item[d)] Compute $\det(D)$, where 
        \[
        D = \begin{bmatrix} 5 & 2 & -3 & 2 \\ 1 & 0 & 4 & 1 \\ 0 & 5 & 0 & -2 \\ 1 & 1 & 5 & -1 \end{bmatrix}
        \]

        To simplify calculations, we expand along the third row and after going through all the same steps we get:
        $\det(D) = 195$. 
        Let me know if there if you encounter any problems when doing the calculations, but there is nothing new here, so maybe consider rereading the solution for the previous item.
        

    \end{enumerate}

    
\end{sol}

